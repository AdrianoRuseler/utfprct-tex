%%%% RESUMO
%%
%% Apresentação concisa dos pontos relevantes de um texto, fornecendo uma visão rápida e clara do conteúdo e das conclusões do trabalho.

\begin{resumoutfpr}%% Ambiente resumoutfpr
O resumo deve ser redigido na terceira pessoa do singular, com verbo na voz ativa, não ultrapassando uma página (de 150 a 500 palavras, segundo a ABNT NBR 6028), evitando-se o uso de parágrafos no meio do resumo, assim como fórmulas, equações e símbolos. Iniciar o resumo situando o trabalho no contexto geral, apresentar os objetivos, descrever a metodologia adotada, relatar a contribuição própria, comentar os resultados obtidos e finalmente apresentar as conclusões mais importantes do trabalho. As palavras-chave devem aparecer logo abaixo do resumo, antecedidas da expressão Palavras-chave. Para definição das palavras-chave (e suas correspondentes em inglês no \textit{abstract}) consultar em Termo tópico do Catálogo de Autoridades da Biblioteca Nacional, disponível em: \url{http://acervo.bn.br/sophia_web/index.html}.
\end{resumoutfpr}
