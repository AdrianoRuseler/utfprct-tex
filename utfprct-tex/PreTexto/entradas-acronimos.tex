%%%% LISTA DE ABREVIATURAS, SIGLAS E ACRÔNIMOS
%%
%% Relação, em ordem alfabética, das abreviaturas (representação de uma palavra por meio de alguma(s) de sua(s) sílaba(s) ou
%% letra(s)), siglas (conjunto de letras iniciais dos vocábulos e/ou números que representa um determinado nome) e acrônimos
%% (conjunto de letras iniciais dos vocábulos e/ou números que representa um determinado nome, formando uma palavra pronunciável).
%%
%% Este arquivo para definição de abreviaturas, siglas e acrônimos é utilizado com a opção "glossaries" (pacote)

%% Como referenciar: 
%% \gls{lp} = Linear Programming (LP)  (First use)
%% \gls{lp} = LP (Next uses)
%% \glspl{lp} = LPs
%% \glsentrytext{lp} = Linear Programming    (recommended for chapter/section/....)
%% \glsentrylong{lp} = Linear Programming
%% \glsentryshort{lp} = LP

%% Para acrônimos também funciona:  
%% \acrlong{lp} = Linear Programming
%% \acrshort{lp} = LP

%% Abreviaturas: \abreviatura{rótulo}{representação}{definição}

\abreviatura{art.}{art.}{Artigo}
\abreviatura{cap.}{cap.}{Capítulo}
\abreviatura{sec.}{sec.}{Seção}

%% Siglas: \sigla{rótulo}{representação}{definição}

\sigla{abnt}{ABNT}{Associação Brasileira de Normas Técnicas}
\sigla{cnpq}{CNPq}{Conselho Nacional de Desenvolvimento Científico e Tecnológico}
\sigla{eps}{EPS}{\textit{Encapsulated PostScript}}
\sigla{pdf}{PDF}{Formato de Documento Portátil, do inglês \textit{Portable Document Format}}
\sigla{ps}{PS}{\textit{PostScript}}
\sigla{utfpr}{UTFPR}{Universidade Tecnológica Federal do Paraná}

%% Acrônimos: \acronimo{rótulo}{representação}{definição}

\acronimo{gimp}{Gimp}{Programa de Manipulação de Imagem GNU, do inglês \textit{GNU Image Manipulation Program}}
