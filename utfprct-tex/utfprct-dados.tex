%% UTFPRCT-TEX, v1.0 wmeira on 2020/06/01
%% Copyright (C) 2020- by William H. T. Meira
%%
%% Modified version of project 'utfprpgtex' maintained 
%% by Luiz E. M. Lima
%%
%% This work may be distributed and/or modified under the
%% conditions of the LaTeX Project Public License, either version 1.3
%% of this license or (at your option) any later version.
%% The latest version of this license is in
%%   http://www.latex-project.org/lppl.txt
%% and version 1.3 or later is part of all distributions of LaTeX
%% version 2005/12/01 or later.
%%
%% This work has the LPPL maintenance status `maintained'.
%%
%% The Current Maintainer of this work is William H. T. Meira.
%%
%% This project consists mainly of files: 'utfprct.cls', 'utfprct.tex', 
%% 'utfprct-dados.tex' 
%% 
%% The 'abntex2-alf.bst' and 'abntex2-num.bst' files are slightly
%% modified versions of the bibtex styles from abntex2 (v.1.9.7)
%% package to suit NBR6023/2018 (not yet implemented). Complementary,
%% 'abntex2-alf-en.bst' and 'abntex2-num-en.bst' are english versions 
%% of the respective bibtex styles.
%%
%% Contribute to improve this project (github repo):
%% https://github.com/wmeira/utfprct-tex

%%%%%%%%%%%%%%%%%%%%%%%%%%%%%%%%%%%%%%%%%%%%%%%
%%%%%%%%%%%%%%%%%%%%%%%%%%%%%%%%%%%%%%%%%%%%%%%
%% Tutorial do Documento de Dados 
%%%%%%%%%%%%%%%%%%%%%%%%%%%%%%%%%%%%%%%%%%%%%%%
%%
%% O 'utfprct-dados.tex' contém todos as informações do documento, 
%% metadados e outros valores importantes para o preenchimento dos 
%% elementos pré-textuais: capa, folha de rosto, resumo, abstract.
%% 
%% Não é necessário preencher todos os campos, existem campos mais
%% específicos para o tipo de documento sendo elaborado. Quando TCC,
%% por exemplo, não é necessário definir dados do programa de 
%% pós-graduação. Todos os dados inseridos estarão disponíveis para
%% inserção no tex usando o padrão '\imprimir + nomedodadominusculo'. 
%% Por exemplo, para imprimir o tipo do documento ('\TipoDeDocumento'),
%% usar '\imprimirtipodedocumento'
%%
%% É possível customizar a descrição do documento na folha de rosto. 
%% Exemplos de descrição são fornecidos para guiar a escrita do texto,
%% em que pode-se utilizar os dados já definidos com o padrão descrito.
%%
%% Existe a possibilidade de inserir dados de uma instituição de 
%% cotutela quando se aplicar. 
%%
%% Os dados da ficha catalográfica são fornecidos pela biblioteca
%% e, na maioria das vezes, apenas anexa-se a folha digitalizada
%% na região definida dos elementos pré-textuais.
%%
%% A folha de aprovação é, na maioria das vezes, fornecido 
%% digitalmente pelo departamento ou pelo orientador após a defesa e 
%% deverá ser anexada ao documento na região definida dos elementos 
%% pré-textuais. Recomenda-se apenas anexar a folha de aprovação sem
%% precisar alterar os dados específicos aqui presentes, pois foram
%% criados originalmente para o template da folha de aprovação da
%% UTFPR-PG, presente no projeto base, e foram apenas mantidos.
%%
%%%%%%%%%%%%%%%%%%%%%%%%%%%%%%%%%%%%%%%%%%%%%%%
%%%%%%%%%%%%%%%%%%%%%%%%%%%%%%%%%%%%%%%%%%%%%%%

%%%%%%%%%%%%%%%%%%%%%%%%%%%%%%%%%%%%%%%%%%%%%%%
%% Informações do Documento
%%%%%%%%%%%%%%%%%%%%%%%%%%%%%%%%%%%%%%%%%%%%%%%

%% Tipo de documento: "Tese", "Dissertação", "Trabalho de Conclusão de Curso"
\TipoDeDocumento{Tese}%

%% [abstract] Document type: "Thesis", "Dissertation", "Bachelor Thesis"
\DocumentType{Thesis}%

%% Nível de formação: "Doutorado", "Mestrado", "Bacharelado"
\NivelDeFormacao{Doutorado}%

%% [abstract] Formation level: "Doctorate", "PhD", "Master's Degree", "Bachelor's Degree"
\FormationLevel{Doctorate}%

%% Título pretendido: "Doutor(a)", "Mestre(a)" ou "Bacharel(a)"
\TituloPretendido{Doutor(a)}%

%% Título do documento
\TituloDoDocumento{%%
Título do presente trabalho acadêmico: subtítulo do presente trabalho acadêmico
}%

%% [abstract] Document title
\DocumentTitle{%%
Title of this academic work: subtitle of this academic work
}%

%% Título em múltiplas linhas na capa, folha de rosto e termo de aprovação
%% Use o comando \par para indicar a quebra de linha
\TituloEmMultiplasLinhas{%%
Título do presente trabalho acadêmico: \par 
subtítulo do presente trabalho acadêmico
}%

%% Data da defesa
\Dia{1}%% Dia (opcional: usado na ficha catalográfica apenas)
\MesPorExtenso{janeiro}%% mês por extenso (opcional: usado na ficha catalográfica apenas)
\Ano{2020}%% Ano

%% Palavras-chave e keywords
\NumeroDePalavrasChave{5}%% Número de palavras-chave (máximo 5)
\PalavraChaveA{Palavra-chave A}%% Palavra-chave A
\PalavraChaveB{Palavra-chave B}%% Palavra-chave B
\PalavraChaveC{Palavra-chave C}%% Palavra-chave C
\PalavraChaveD{Palavra-chave D}%% Palavra-chave D
\PalavraChaveE{Palavra-chave E}%% Palavra-chave E
\NumeroDeKeywords{5}%% Número de keywords (máximo 5)
\KeywordA{Keyword A}%% Keyword A
\KeywordB{Keyword B}%% Keyword B
\KeywordC{Keyword C}%% Keyword C
\KeywordD{Keyword D}%% Keyword D
\KeywordE{Keyword E}%% Keyword E

%%%%%%%%%%%%%%%%%%%%%%%%%%%%%%%%%%%%%%%%%%%%%%%
%% Informação do Autor(a) ou Autores(as) (TCC)
%%%%%%%%%%%%%%%%%%%%%%%%%%%%%%%%%%%%%%%%%%%%%%%

%%% Autor(a)
%% Usado para citação: "\SobrenomeDoAutor, PrenomeDoAutor" (ex: "Doe, John" ou "Doe, J.")
\NomeDoAutor{Nome do(a) Autor(a)}%% Nome completo do(a) autor(a)
\SobrenomeDoAutor{Último Nome}%% Último nome do(a) autor(a)
\PrenomeDoAutor{Nome do(a) Autor(a) Sem Último}%% Nome do(a) autor(a) sem último nome

%%% Autor(a) 2 (opcional)
%% *Considera apenas se "\TipoDeDocumento" == "Trabalho de Conclusão de Curso"
\AtribuiAutorDois{true}%% Insere ou remove autor(a) 2: "true" ou "false"
\NomeDoAutorDois{Nome do(a) Autor(a) 2}%% Nome completo do(a) autor(a) 2
\SobrenomeDoAutorDois{Último Nome}%% Último nome do(a) autor(a) 2
\PrenomeDoAutorDois{Nome do(a) Autor(a) 2 Sem Último}%% Nome do(a) autor(a) 2 sem último nome

%%% Autor(a) 3 (opcional)
%% *Considera apenas se "\TipoDeDocumento" == "Trabalho de Conclusão de Curso"
\AtribuiAutorTres{true}%% Insere ou remove autor(a) 3: "true" ou "false"
\NomeDoAutorTres{Nome do(a) Autor(a) 3}%% Nome completo do(a) autor(a) 3
\SobrenomeDoAutorTres{Último Nome}%% Último nome do(a) autor(a) 3
\PrenomeDoAutorTres{Nome do(a) Autor(a) 3 Sem Último}%% Nome do(a) autor(a) 3 sem último nome

%%%%%%%%%%%%%%%%%%%%%%%%%%%%%%%%%%%%%%%%%%%%%%%%%%
%% Informações do Orientador(a) e Coorientador(a)
%%%%%%%%%%%%%%%%%%%%%%%%%%%%%%%%%%%%%%%%%%%%%%%%%%

%% Orientador(a)
%% Usado para citação: "\SobrenomeDoOrientador, PrenomeDoOrientador" (ex: "Doe, John" ou "Doe, J.")
\AtribuicaoOrientador{Orientador(a)}%% Atribuição "Orientador(a)"
\TituloDoOrientador{Prof(a). Dr(a).}%% Título do(a) orientador(a)
\NomeDoOrientador{Nome Completo do(a) Orientador(a)}%% Nome completo do(a) orientador(a)
\SobrenomeDoOrienador{Último Nome}%% Último nome do(a) orientador(a)
\PrenomeDoOrientador{Nome do(a) Orientador(a) Sem Último}%% Nome do(a) orientador(a) sem último nome

%% Coorientador(a) (opcional)
%% Usado para citação: "\SobrenomeDoCoorientador, PrenomeDoCoorientador" (ex: "Doe, John" ou "Doe, J.")
\AtribuiCoorientador{true}%% Insere ou remove o(a) coorientador(a): "true" ou "false"
\AtribuicaoCoorientador{Coorientador(a)}%% Atribuição "Coorientador(a)"
\TituloDoCoorientador{Prof(a). Dr(a).}%% Título do(a) coorientador(a)
\NomeDoCoorientador{Nome Completo do(a) Coorientador(a)}%% Nome completo do(a) coorientador(a)
\SobrenomeDoCoorienador{Último Nome}%% Último nome do(a) coorientador(a)
\PrenomeDoCoorientador{Nome do(a) Coorientador(a) Sem Último}%% Nome do(a) coorientador(a) sem último nome

%%%%%%%%%%%%%%%%%%%%%%%%%%%%%%%%%%%%%%%%%%%%%%%%%%
%% Informações da Instituição
%%%%%%%%%%%%%%%%%%%%%%%%%%%%%%%%%%%%%%%%%%%%%%%%%%

%% Nome da instituição
\Instituicao{Universidade Tecnológica Federal do Paraná}

%% [abstract] Institution name (*nome sem traduzir é o recomendado para docs. da UTFPR)
\Institution{Universidade Tecnológica Federal do Paraná}

%% Sigla da Instituição
\SiglaInstituicao{UTFPR}

%% Nome da cidade (câmpus)
\Cidade{Curitiba}

%% Diretoria: "Graduação e Educação Profissional" ou "Pesquisa e Pós-Graduação" (opcional: usado na ficha catalográfica)
\Diretoria{Pesquisa e Pós-Graduação}

%% Nome do departamento ou da coordenação (opcional: mais comum no Bacharelado: Departamento de Informática)
\Departamento{Nome do Departamento ou da Coordenação}

%% Sigla do departamento (opcional, ex: DAINF, DAMEC, DAMAT...)
\SiglaDepartamento{DPT}

%% Nome do curso bachalerado ou pós-graduação (PPG) (ex: "Engenharia de Computação",  "Engenharia El{\'e}trica e Inform{\'a}tica Industrial") 
\Curso{Nome do Curso}

%% [abstract] Course name
\Course{Course Name}

%% Programa ou nome do curso completo (capa)
%% "Bachalerado em Engenharia de Computação"
%% "Programa de Pós-Graduação em Engenharia Elétrica e Informática Industrial"
\Programa{Programa de Pós-Graduação em \imprimircurso}

%% Sigla do programa de pós-graduação (opcional, ex: CPGEI, PPGCA)
\SiglaDoPPG{PPG}

%% Nome da área de concentração
\AreaDeConcentracao{Nome da Área de Concentração}

%%%%%%%%%%%%%%%%%%%%%%%%%%%%%%%%%%%%%%%%%%%%%%%%%%%%%%
%% Informações de Cotutela (Duplo Grau) (opcional)
%%%%%%%%%%%%%%%%%%%%%%%%%%%%%%%%%%%%%%%%%%%%%%%%%%%%%%

%% Insere dados de cotutela: "true" ou "false"
\AtribuiCotutela{false}

%% Nome da instituição de cotutela
\InstituicaoCotutela{Universidade Da Cotutela}

%% [abstract] Institution name
\InstitutionCotutela{Double Degree University}

%% Sigla da instituição de cotutela
\SiglaInstituicaoCotutela{UC}

%% Nome do curso bachalerado ou pós-graduação (PPG) na instituição de cotutela (ex: "Engenharia de Computação",  "Engenharia El{\'e}trica e Inform{\'a}tica Industrial")
\CursoCotutela{Nome do Curso Cotutela}

%% [abstract] Course name 
\CourseCotutela{Second Degree Course}

%% Nome do departamento ou da coordenação da inst. de cotutela (mais comum no Bacharelado: Departamento de Informática)
\DepartamentoCotutela{Nome do Departamento ou da Coordenação}

%% Sigla do departamento da inst. cotutela (ex: DAINF, DAMEC, DAMAT...)
\SiglaDepartamentoCotutela{DPT-EXT}

%% Programa ou nome do curso completo na inst. cotutela (capa)
%% "Bachalerado em Engenharia de Computação"
%% "Programa de Pós-Graduação em Engenharia Elétrica e Informática Industrial"
\ProgramaCotutela{Programa de Doutoral em \imprimircursocotutela}

%% Sigla do programa externo de pós-graduação
\SiglaDoPPGCotutela{PPG-EXT}

%% Nome da área de concentração na instituição de cotutela
\AreaDeConcentracaoCotutela{Nome da Área de Concentração}

%% Nível de formação que será fornecido na referência do doc.
\NivelDeFormacaoResumo{Duplo doutorado}
\FormationLevelAbstract{Double PhD}

%% Informacoes do orientador(a) na instituição de cotutela

%% Orientador(a) da instituição de cotutela
%% Usado para citação: "\SobrenomeDoOrientador, PrenomeDoOrientador" (ex: "Doe, John" ou "Doe, J.")
\AtribuicaoOrientadorCotutela{Orientador(a)}%% Atribuição "Orientador(a)"
\TituloDoOrientadorCotutela{Prof(a). Dr(a).}%% Título do(a) orientador(a)
\NomeDoOrientadorCotutela{Nome Completo do(a) Orientador(a)}%% Nome completo do(a) orientador(a)
\SobrenomeDoOrienadorCotutela{Último Nome}%% Último nome do(a) orientador(a)
\PrenomeDoOrientadorCotutela{Nome do(a) Orientador(a) Sem Último}%% Nome do(a) orientador(a) sem último nome

%% Coorientador(a) da instituição de cotutela
%% Usado para citação: "\SobrenomeDoCoorientador, PrenomeDoCoorientador" (ex: "Doe, John" ou "Doe, J.")
\AtribuiCoorientadorCotutela{true}%% Insere ou remove o(a) coorientador(a) da cotutela: "true" ou "false"
\AtribuicaoCoorientadorCotutela{Coorientador(a)}%% Atribuição "Coorientador(a)"
\TituloDoCoorientadorCotutela{Prof(a). Dr(a).}%% Título do(a) coorientador(a)
\NomeDoCoorientadorCotutela{Nome Completo do(a) Coorientador(a)}%% Nome completo do(a) coorientador(a)
\SobrenomeDoCoorienadorCotutela{Último Nome}%% Último nome do(a) coorientador(a)
\PrenomeDoCoorientadorCotutela{Nome do(a) Coorientador(a) Sem Último}%% Nome do(a) coorientador(a) sem último nome

%%%%%%%%%%%%%%%%%%%%%%%%%%%%%%%%%%%%%%%%%%%%%%%%%%
%% Folha de Rosto
%%%%%%%%%%%%%%%%%%%%%%%%%%%%%%%%%%%%%%%%%%%%%%%%%%

%% Descrição do documento na folha de rosto (exemplos)
\DescricaoDoDocumento{
%\imprimirtipodedocumento\ apresentada ao Programa de Pós-Graduação em \imprimircurso\ (\imprimirsigladoppg) da \imprimirinstituicao\ (\imprimirsiglainstituicao)\ como requisito à obtenção do título de ``Doutor em Ciências''. Área de Concentração: \imprimirareadeconcentracao.%
\imprimirtipodedocumento\ apresentado(a) como requisito parcial à obtenção do título de \imprimirtitulopretendido\ em \imprimircurso, do \imprimirppgoudepartamento, da \imprimirinstituicao.
}

%% Insere ou remove descrição da cotutela (extra) na folha de rosto: "true" ou "false". 
%% Se "true", a descrição do documento será colocada na folha de rosto, logo abaixo do orientador(a) e coorientador(a) da primeira inst. e depois o orientador(a) e coorientador(a) da inst. de cotutela. 
%% Se "false", os nomes do orientador(a) e coorientador(a) aparecerão logo abaixo do orientador(a) da primeira instituição, sem uma descrição extra. Neste caso, recomenda-se revisar a "\DescricaoDoDocumento" para contemplar ambas as instituições.   
\AtribuiDescricaoCotutela{true}

%% Segunda Descricao da Inst. de Cotutela na folha de rosto (exemplos)
\DescricaoDoDocumentoCotutela{
\imprimirtipodedocumento\ apresentado(a) como requisito parcial à obtenção do título de \imprimirtitulopretendido\ em \imprimircursocotutela, do \imprimirppgoudepartamentocotutela, da \imprimirinstituicaocotutela.  
%\imprimirtipodedocumento\ apresentada à Comissão de Acompanhamento de Tese do Programa Doutoral em \imprimircursocotutela\ do \imprimirinstituicaocotutela\ (\imprimirsiglainstituicaocotutela) como requisito à obtenção de grau de \imprimirtitulopretendido\ na área de concentração \imprimirareadeconcentracaocotutela.
}

%%%%%%%%%%%%%%%%%%%%%%%%%%%%%%%%%%%%%%%%%%%%%%%%%%
%% Ficha Catalográfica* (opcional)
%%%%%%%%%%%%%%%%%%%%%%%%%%%%%%%%%%%%%%%%%%%%%%%%%%

%% *Pode ser usado como placeholder, porém para entrega deve-se inserir a ficha catolográfica digitalizado (PDF) pela biblioteca da UTFPR.

\NumeroDaPublicacao{00/\imprimirano}%% Número da publicação - Fornecido pela biblioteca
\CDDOuCDU{CDD 000.00}%% Classificação Decimal Dewey (CDD) ou Classificação Decimal Universal (CDU) - Fornecida pela biblioteca

\TituloDaFichaCatalografica{%% Título da ficha catalográfica
  Ficha catalográfica elaborada pelo Departamento de Biblioteca da \par \imprimirinstituicao, Câmpus \imprimircidade \par n.
  \imprimirnumerodapublicacao
}

%%%%%%%%%%%%%%%%%%%%%%%%%%%%%%%%%%%%%%%%%%%%%%%%%%
%% Folha de aprovação (Formato UTFPR-PG)* (opcional)
%%%%%%%%%%%%%%%%%%%%%%%%%%%%%%%%%%%%%%%%%%%%%%%%%%

%% *Pode ser usado como placeholder, porém para entrega final prefere-se a folha (termo) de aprovação digitalizado (PDF) fornecido pelo departamento ou orientador.

\NumeroDaTeseOuDissertacao{00/\imprimirano}%% Número da Tese ou Dissertação - Fornecido pelo programa de pós-graduação
\NumeroDaFichaCatalografica{A000}%% Número da ficha catalográfica - Fornecido pela biblioteca

\TituloDoResponsavelTCC{Prof(a). Dr(a).}%% Título do(a) responsável pelos TCC
\NomeDoResponsavelTCC{Nome do(a) Responsável}%% Nome completo do(a) responsável pelos TCC
\AtribuicaoCoordenador{Coordenador(a)}%% Atribuição "Coordenador(a)" do curso
\TituloDoCoordenador{Prof(a). Dr(a).}%% Título do(a) coordenador(a) do curso
\NomeDoCoordenador{Nome do(a) Coordenador(a)}%% Nome completo do(a) coordenador(a) do curso

\TextoDeAprovacao{%% Texto de aprovação
  %% Exemplo de texto de aprovação para Tese ou Dissertação (descomente a próxima linha para utilizá-lo):
  Esta \imprimirtipodedocumento\ foi apresentada às 00:00 de \imprimirdia\ de \imprimirmesporextenso\ de \imprimirano\ como requisito parcial para a obtenção do título de \imprimirtitulopretendido\ em \imprimircurso, na área de concentração em \imprimirareadeconcentracao\ e na linha de pesquisa em (Nome da Linha de Pesquisa), do Programa de Pós-Graduação em \imprimircurso. O(A) candidato(a) foi arguido(a) pela Banca Examinadora composta pelos professores abaixo citados. Após deliberação, a Banca Examinadora considerou o trabalho aprovado.
  %% Exemplo de texto de aprovação para Trabalho de Conclusão de Curso (descomente a próxima linha para utilizá-lo):
  % Este \imprimirtipodedocumento\ foi apresentado em \imprimirdia\ de \imprimirmesporextenso\ de \imprimirano\ como requisito parcial para a obtenção do título de \imprimirtitulopretendido\ em \imprimircurso. O(A) candidato(a) foi arguido(a) pela Banca Examinadora composta pelos professores abaixo assinados. Após deliberação, a Banca Examinadora considerou o trabalho aprovado.
}

\AvisoDeAprovacao{%% Aviso de aprovação
  %% Exemplo de aviso de aprovação para Tese ou Dissertação (descomente a próxima linha para utilizá-lo):
  A Folha de Aprovação assinada encontra-se no \par Departamento de Registros Acadêmicos da UTFPR -- Câmpus \imprimircidade
  %% Exemplo de aviso de aprovação para Trabalho de Conclusão de Curso (descomente a próxima linha para utilizá-lo):
  % -- O Termo de Aprovação assinado encontra-se na Coordenação do Curso --
}

%% Banca examinadora: 3 membros (Trabalho de Conclusão de Curso ou Dissertação); 5 a 7 membros (Tese)
\MembroAIgualOrientador{true}%% Insere ou remove o membro A igual ao(à) orientador(a): "true" ou "false"
\MembroA{Nome do Membro A}%% Nome completo do membro A - Presidente (automático se orientador(a))
\TituloDoMembroA{Prof(a). Dr(a).}%% Título do membro A - Presidente (automático se orientador(a))
\InstituicaoDoMembroA{Instituição do Membro A}%% Nome da instituição do membro A - Presidente (automático se orientador(a))
\MembroB{Nome do Membro B}%% Nome completo do membro B
\TituloDoMembroB{Prof(a). Dr(a).}%% Título do membro B
\InstituicaoDoMembroB{Instituição do Membro B}%% Nome da instituição do membro B
\MembroC{Nome do Membro C}%% Nome completo do membro C
\TituloDoMembroC{Prof(a). Dr(a).}%% Título do membro C
\InstituicaoDoMembroC{Instituição do Membro C}%% Nome da instituição do membro C
\MembroD{Nome do Membro D}%% Nome completo do membro D
\TituloDoMembroD{Prof(a). Dr(a).}%% Título do membro D
\InstituicaoDoMembroD{Instituição do Membro D}%% Nome da instituição do membro D
\MembroE{Nome do Membro E}%% Nome completo do membro E
\TituloDoMembroE{Prof(a). Dr(a).}%% Título do membro E
\InstituicaoDoMembroE{Instituição do Membro E}%% Nome da instituição do membro E
\AtribuiMembroF{false}%% Insere ou remove o Membro F: "true" ou "false"
\MembroF{Nome do Membro F}%% Nome completo do membro F
\TituloDoMembroF{Prof(a). Dr(a).}%% Título do membro F
\InstituicaoDoMembroF{Instituição do Membro F}%% Nome da instituição do membro F
\AtribuiMembroG{false}%% Insere ou remove o Membro G: "true" ou "false"
\MembroG{Nome do Membro G}%% Nome completo do membro G
\TituloDoMembroG{Prof(a). Dr(a).}%% Título do membro G
\InstituicaoDoMembroG{Instituição do Membro G}%% Nome da instituição do membro G
